\documentclass[]{article}

\usepackage{subfiles}
\usepackage{amssymb}
\usepackage{amsthm}
\usepackage{mathtools}
\usepackage{nicefrac}

\DeclareMathOperator{\GL}{GL}
\DeclareMathOperator{\SL}{SL}
\DeclareMathOperator{\vspan}{span}

\newcommand {\CC}{\mathbb{C}}
\newcommand {\RR}{\mathbb{R}}
\newcommand {\FF}{\mathbb{F}}
\newcommand {\NN}{\mathbb{N}}
\newcommand {\QQ}{\mathbb{Q}}
\newcommand {\EE}{\mathbb{E}}
\newcommand {\PP}{\mathbb{P}}

\newtheorem{theorem}{Theorem}
\newtheorem{proposition}{Proposition}
\theoremstyle{definition}
\newtheorem{definition}{Definition}
\newtheorem{example}{Example}
\newtheorem{note}{Note}
\newtheorem{notation}{Notation}
%opening
\title{Mathematical Notebook}
\author{}
\date{}

\begin{document}

\maketitle
\begin{notation} We will write
	\begin{enumerate}
		\item $\mathcal{M}_{n \times m}(\FF)$ for the set of $n$ by $m$ matrices with entries in the field $\FF$.
		\item $\GL(n, \RR) = \{ M \in \mathcal{M}_{n \times n} (\RR): \det M \neq 0 \}$.
		\item $\SL(n, \RR) = \{ M \in \mathcal{M}_{n \times n} (\RR): \det M = 1  \}$.
		\item The transpose of $M \in \mathcal{M}_{n \times m} (\RR) $ is denoted by $M^\intercal$.
	\end{enumerate}
\end{notation}

\section{Review}
\subfile{review}


\end{document}
